\documentclass[]{scrartcl}

\usepackage[T1]{fontenc}
\usepackage[utf8]{inputenc}
\usepackage[ngerman]{babel}
\usepackage[tmargin=2cm,lmargin=3cm,rmargin=3cm,bmargin=1.5cm]{geometry}

%opening
\title{Gemüse-Fleischsuppe}
\author{Noemi Borer}

%custom commands
\newcommand{\zutat}[1]{\textbf{#1}}
\setkomafont{section}{\huge\underline}


\begin{document}

\maketitle

\section*{Zutaten}
\begin{description}
	\item [Butter]
	\item [Knoblauch] gepresst
	\item [Zwiebel] Klein gehackt
	\item [Rüebli] In groben Stücke geschnitten
	\item [Lauch] Klein geschnitten
	\item [Sellerie] Klein geschnitten
	\item [Gewürze] Salz oder Bouillon
	\item [Fleisch] Siedfleisch durchzogen
	\item [Markknochen]
	
\end{description}
\section*{Zubereitung}
	\begin{enumerate}
		\item \zutat{Gemüse} zusammen mit \zutat{Butter} andämpfen
		\item Mit \emph{warmem} \zutat{Wasser} auffüllen
		\item Nach eigenem Wunsch \zutat{Gewürze} dazugeben
		\item \zutat{Fleisch} und \zutat{Markknochen} ins Wasser geben
		\item ca 40--60 Minuten köcheln lassen
	\end{enumerate}
\subsection*{Beachten}
\begin{itemize}
\item Beim Andünsten gut rühren 
\item Wassermenge der Zutaten anpassen
\item Fleisch separat im Kühlschrank aufbewahren für den nächsten Tag
\end{itemize}
\end{document}
